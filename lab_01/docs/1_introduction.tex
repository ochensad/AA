\section*{\large Введение}
\addcontentsline{toc}{chapter}{Введение}
	Термин «динамическое программирование» впервые был использован в середине 20-го века. С тех пор им ежедневно пользуется каждый пользователь Интернета, даже не подозревая об этом.
	\par Динамическое программирование -- это способ решения сложных задач путём разбиения их на более простые подзадачи. Так, например, одной из задач динамического программирования является задача сравнения двух строк. Советский математик Владимир Иосифович Левенштейн ввел понятие «рекадционного расстояния» между двумя строковыми последовательностями и разработал алгоритм для его нахождения. 
	
	Расстояние Левенштейна применяется в теории информации и компьютерной лингвистике для:

	\begin{enumerate}
		\item автоматического исправления ошибок в слове;
		\item сравнение введёных слов со словарями в поисковых запросах;
		\item в биоинформатике для сравнения генов, хромосом и белков.
	\end{enumerate}

	Цель лабораторной работы:\\
	изучение и исследование особенностей задач динамического программирования.

	Задачи лабораторной работы:
	\begin{enumerate}
		\item выбрать инструменты для замера процессорного времени выполнения реализации алгоритмов;
		\item изучить алгоритмы нахождения расстояния Левенштейна и Дамерау-Левенштейна;
		\item реализовать:
		\begin{enumerate}
			\item нерекурсивный метод поиска расстояния Левенштейна;
			\item нерекурсивный метод поиска расстояния Дамерау-Левенштейна;
			\item рекурсивный метод поиска расстояния Дамерау-Левенштейна;
			\item рекурсивный с кэшированием метод поиска расстояния Дамерау-Левенштейна;
		\end{enumerate}
		\item замерить процессорное время и потребленную память для всех реализованных алгоритмов;
		\item провести анализ работы программы по времени и по памяти, выяснить влияющие на них характеристики.
	\end{enumerate}
\newpage