\chapter{ Аналитический раздел}
\label{cha:analytical}
    \section{Расстояние Левенштейна}

	Расстояние Левенштейна - это минимальное необходимое количество редакторских операций (вставки, замены, удаления) для преобразования одной строки в другую. Все три операции обладают штрафом 1. Операция совпадения обладает штрафом 0.
	\par\textbf{Обозначение редакторских операций:} 
	\begin{itemize}
		\item I (insert) -- вставка символа;
		\item R (replace) -- замена символа;
		\item D (delete) -- удаление символа;
		\item M (match) -- совпадение символов.
	\end{itemize}
	
	Задача по нахождению Расстояния Левенштейна заключается в поиске минимального количества операций для превращения одной строки в другую.

    Пусть $s_{1}$ и $s_{2}$ — две строки (длиной N и M соответственно) в некотором алфавите V,
    тогда расстояние Левенштейна можно подсчитать по рекуррентной формуле (\ref{formula:Levenshtein}):

    \begin{equation}\label{formula:Levenshtein}
        	D_{s_1,s_2}(i, j) = \begin{cases}
        	\max(i, j), &\text{если }\min(i, j) = 0,\\
        	\min \lbrace \\
        	\qquad D_{s_1,s_2}(i, j-1) + 1,\\
        	\qquad D_{s_1,s_2}(i-1, j) + 1,\\
        	\qquad D_{s_1,s_2}(i-1, j-1) + m(a[i], b[j])
        		\rbrace
    \end{cases},
    \end{equation}\par где \begin{math}i,j\end{math} -- индексы i-го и j-го символов строк $s_{1}$ и $s_{2}$ соответственно.

	\section{Нерекурсивный алгоритм нахождения расстояния Левенштейна с использованием матрицы}
	
	
	Данный алгоритм заключается в решении задачи с использованием матрицы расстояний, которая построчно заполняется последовательно вычисляемыми $D(i, j)$. 
	
		В Таблице \ref{table:example:Levenshtein} минимальное расстояние между
	словом "кот" и "скат" равно 2. Последовательность редакторских операций,
	которая привела к ответу - IMRM.
	
	\begin{table}[h]
		\caption{Пример работы преобразования слова "кот" в "скат"}
		\centering
		\begin{tabular}{|c|c|c|c|c|c|}
			\hline
			& $\lambda$ & С & К & А & Т \\ \hline
			$\lambda$ & 0 & 1 & 2 & 3 & 4 \\ \hline
			К & 1 & 1 & 1 & 2 & 3 \\ \hline
			О & 2 & 2 & 2 & 2 & 3 \\ \hline
			Т & 3 & 3 & 3 & 3 & \cellcolor[HTML]{FFCCC9}2 \\ \hline
		\end{tabular}
		\label{table:example:Levenshtein}
	\end{table}
    \section{Расстояние Дамерау-Левенштейна}  
    Расстояние Дамерау-Левенштейна является модификацией расстояния Левенштейна:
    к операциям вставки, удаления и замены символов, добавлена операция транспозиции (перестановки
    двух соседних символов) (X - exchange). Операция транспозиции возможна, если символы попарно совпадают.

    Пусть $s_{1}$ и $s_{2}$ — две строки (длиной N и M соответственно) в некотором алфавите V,
    тогда расстояние Дамерау-Левенштейна можно подсчитать по рекуррентной формуле (\ref{formula:DamerauLevenshtein}):

    \begin{equation}\label{formula:DamerauLevenshtein}
	D_{a,b}(i, j) = \begin{cases}
		\max(i, j), &\text{если }\min(i, j) = 0,\\
		\min \lbrace \\
		\qquad D_{a,b}(i, j-1) + 1,\\
		\qquad D_{a,b}(i-1, j) + 1,\\
		\qquad D_{a,b}(i-1, j-1) + m(a[i], b[j]), &\text{иначе}\\
		\qquad \left[ \begin{array}{cc}D_{a,b}(i-2, j-2) + 1, &\text{если }i,j > 1;\\
			\qquad &\text{}a[i] = b[j-1]; \\
			\qquad &\text{}b[j] = a[i-1]\\
			\qquad \infty, & \text{иначе}\end{array}\right.\\
		\rbrace
	\end{cases},
\end{equation} \par где \begin{math}i,j\end{math} -- индексы i-го и j-го символов строк $s_{1}$ и $s_{2}$ соответственно.

	\section{Нерекурсивный алгоритм нахождения расстояния Дамерау-Левенштейна с использованием матрицы}
	
	
	Данный алгоритм, также как и алгоритм Левенштейна, заключается в решении задачи с использованием матрицы расстояний, которая построчно заполняется последовательно вычисляемыми $D(i, j)$. 
	
	
		В Таблице \ref{table:example:DamLevenshtein} минимальное расстояние между
	словом "тест" и "тетс" равно 1. Последовательность редакторских операций,
	которая привела к ответу - MMX.
	
	\begin{table}[h]
		\caption{Пример работы преобразования слова "тетс" в "тест"}
		\centering
		\begin{tabular}{|c|c|c|c|c|c|}
			\hline
			& $\lambda$ & Т & Е & Т & С \\ \hline
			$\lambda$ & 0 & 1 & 2 & 3 & 4 \\ \hline
			Т & 1 & 0 & 1 & 2 & 3 \\ \hline
			Е & 2 & 1 & \cellcolor[HTML]{AFEEEE}0 & 1 & 2 \\ \hline
			С & 3 & 2 & 1 & \cellcolor[HTML]{AFEEEE}1 & \cellcolor[HTML]{AFEEEE}2 \\ \hline
			Т & 4 & 3 & 2 & \cellcolor[HTML]{AFEEEE}2 & \cellcolor[HTML]{FFCCC9}1\\ \hline
		\end{tabular}
		\label{table:example:DamLevenshtein}
	\end{table}
	
	Голубым и красным цветом выделены клетки, задействованные в операции обмена X.

	\section{Рекурсивный алгоритм нахождения расстояния Дамерау-Левенштейна}	


	Суть рекурсивного алгоритма заключается в реализации формулы \ref{formula:DamerauLevenshtein}.
	
	\section{Рекурсивный алгоритм нахождения расстояния Дамерау-Левенштейна с использованием матрицы (кэшированием)}
		
		
	Для оптимизации рекурсивного алгоритма нахождения расстояния Дамерау-Левенштейна допустимо добавить матрицу для хранения значений $D(i, j)$ для того, чтобы не вычислять их заново раз за разом. Таким образом, при обработке ещё не затронутых данных, результат нахождения расстояния будет занесён в так называемую матрицу расстояний. В ином случае, если для рассматриваемого случая информация о расстоянии уже имеется в матрице, алгоритм будет переходить к следующему шагу.


	
	\section*{Вывод}

	
	Для каждого рассмотренного алгоритма имеется некоторая рекуррентная формула, что даёт возможность изучить как рекурсивные, так и итеративные реализации алгоритмов. Для оптимизации рекурсивных алгоритмов в рассмотрение вводится матрица, в которую записываются все промежуточные вычисленные значения. Эта же матрица применяется и при реализации итеративных алгоритмов.
\newpage