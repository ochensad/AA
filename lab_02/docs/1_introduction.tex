\section*{\large Введение}
\addcontentsline{toc}{chapter}{Введение}

\par Умножение матриц -- одна из основных операций над матрицами. Матрица, получаемая в результате операции умножения, называется произведением матриц. Элементы новой матрицы получаются из элементов старых матриц в соответствии с правилами.
\par Матрицы \begin{math}A\end{math} и \begin{math}B\end{math} могут быть перемножены, если они совместимы в том смысле, что число столбцов матрицы \begin{math}A\end{math} равно числу строк \begin{math}B\end{math}.
	В данной лабораторной работе рассмотриваются алгоритмы:

	\begin{enumerate}
		\item классическое умножение матриц;
		\item алгоритм Винограда;
		\item улучшенный алгоритм Винограда.
	\end{enumerate}

	Цель лабораторной работы:\\
	реализовать алгоритмы перемножения матриц.

	Задачи лабораторной работы:
	\begin{enumerate}
		\item выбрать инструменты для замера процессорного времени выполнения реализации алгоритмов;
		\item изучить алгоритмы классического перемножения матриц, Винограда с оптимизацией и без;
		\item реализовать:
		\begin{enumerate}
			\item алгоритм классического перемножения матриц;
			\item алгоритм Винограда;
			\item алгоритм Винограда с оптимизацией.
		\end{enumerate}
		\item дать оценку трудоёмкости;
		\item провести замеры процессорного времени работы  и затрачиваемой памяти для всех алгоритмов;
		\item привести краткие рекомендации об особенностях применения оптимизаций.
	\end{enumerate}
\newpage