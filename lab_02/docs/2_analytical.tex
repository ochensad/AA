\chapter{Аналитический раздел}
\label{cha:analytical}
	\par В данном разделе будут рассмотрены алгоритмы классического перемножения матриц и Винограда без оптимизации.
    \section{Классическое умножение матриц}
	\par Матрицей \begin{math}A\end{math} размера \begin{math}[N \cdot M]\end{math} называется прямоугольная таблица чисел, функций или алгебраических выражений, содержащая строк \begin{math}N\end{math}  и \begin{math}M\end{math}  столбцов. Если число столбцов в первой матрице совпадает с числом строк во второй, то эти две матрицы можно перемножить. У произведения будет столько же строк, сколько в первой матрице, и столько же столбцов, сколько во второй. [\ref{bib:3}]
	\par Пусть даны две прямоугольные матрицы \begin{math}A\end{math} и \begin{math}B\end{math} размеров \begin{math}[N \cdot M]\end{math} и \begin{math}[M \cdot K]\end{math} соответственно. В результате произведение матриц \begin{math}A\end{math} и \begin{math}B\end{math} получим матрицу \begin{math}C\end{math} размера \begin{math}[N \cdot K]\end{math}.
	\begin{equation}\label{formula:ClassicMult}
	C_{i,j} = \sum_{l=1}^{M} A_{i,l} \cdot B_{l,j}
	\end{equation},
	где \begin{math}i,j\end{math} -- индексы строки и столбца в матрице \begin{math}C\end{math} соответственно.
	\section{Алгоритм Винограда}
	\par Если посмотреть на результат умножения двух матриц, то видно, что каждый элемент в нем представляет собой скалярное произведение соответствующих строки и столбца исходных матриц. Можно заметить также, что такое умножение допускает предварительную обработку, позволяющую часть работы выполнить заранее.[\ref{bib:4}]
	\par Рассмотрим два вектора \begin{math}V = (v_1, v_2, v_3, v_4)\end{math} и \begin{math}H = (h_1, h_2, h_3, h_4)\end{math}.
	\par Скалярное произведение этих векторов равно:
	\begin{equation}\label{formula:VinogradMult}
	V \cdot H = v_1 \cdot h_1 + v_2 \cdot h_2 + v_3 \cdot h_3 + v_4 \cdot h_4
	\end{equation}
	\par Равенство \ref{formula:VinogradMult} можно переписать в виде:
	\begin{equation}\label{formula:VinogradMult_1}
	V \cdot H = (v_1  + h_2) \cdot (v_2  + h_1) + (v_3  + h_4) \cdot (v_4  + h_3) - v_1 \cdot v_2 - v_3 \cdot v_4 - h_1 \cdot h_2 - h_3 \cdot h_4
	\end{equation}
	\par Выражение в правой части последнего равенства допускает предварительную обработку: его части можно вычислить заранее и запомнить для каждой строки первой матрицы и для каждого столбца второй. Это означает, что над предварительно обработанными элементами нам придется выполнять лишь первые два умножения и последующие пять сложений, а также дополнительно два сложения.
	\section*{Вывод}
\par Были рассмотрены алгоритмы классического перемножения матриц и алгоритм Винограда, основное отличие которых — наличие предварительной обработки, а также количество операций умножения.
\newpage