% % Список литературы при помощи BibTeX
% Юзать так:
%
% pdflatex report
% bibtex report
% pdflatex report

\bibliographystyle{gost780u}

\section*{\large Список источников}
\addcontentsline{toc}{chapter}{Список источников}
\begin{enumerate}
	\item Ассемблерные вставки в AVR-GCC. // [Электронный ресурс]. Режим доступа: \url{https://habr.com/ru/post/275937/} дата обращения: 20.09.2022); \label{bib:1}
	\item C/C++: как измерять процессорное время. // [Электронный ресурс]. Режим доступа: \url{https://habr.com/ru/post/282301/} (дата обращения: 20.09.2022);\label{bib:2}
	\item Matrix multiplication. // [Электронный ресурс]. Режим доступа: 
	\url{https://en.wikipedia.org/wiki/MatrixMultiplication} (дата обращения: 03.10.2022).\label{bib:3}
	\item Умножение матриц по Винограду. // [Электронный ресурс]. Режим доступа: \url{http://www.algolib.narod.ru/Math/Matrix.html} (дата обращения: 03.10.2022). \label{bib:4}
	\item Оптимизация в GCC. // [Электронный ресурс]. Режим доступа: \url{https://tproger.ru/translations/will-it-optimize-gcc/} (дата обращения: 03.10.2022). \label{bib:5}
\end{enumerate}