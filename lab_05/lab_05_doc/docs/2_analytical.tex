\chapter{Аналитический раздел}
\label{cha:analytical}
    \section{Конвеерные вычисления}
    \par Способ организации процесса в качетсве вычислительного конвеера позволяет построить процесс, содержащий несколько независимых этапов [\ref{bib:1}], на нескольких потоках. Выигрыш по времени достигается при выполнении нескольких этапов одновременно за счет параллельной работы ступеней. Для контроля стадии используется три основные метрики, описанные ниже.
    \begin{enumerate}
    	\item \textbf{Время процесса} -- это время, необходимое для выполнения одной стадии;
    	\item \textbf{Время выполнения} -- это время, которое требуется с момента, когда работа была выполнена на предыдущем этапе, до выполнения на текущем;
    	\item \textbf{Время простоя} -- это время, когда никакой работы не происходит и линии простаивают.
    \end{enumerate}

    Для того, чтобы время простоя было минимальным, стадии обработки должны быть одинаковы по времени в пределах погрешности. При возникновении ситуации, в которой время процесса одной из линий больше, чем время в других в $N$ раз, эту линию стоит распаралеллить на $N$ потоков.

\section{Шифр Цезаря}
	\par Шифр цезаря -- это преобразование информации методом замены букв на другие, стоящии от даннх через определенное количество символов в алфавите. Сдвиг на определенное количество символов называется ключом шифрования.[\ref{bib:2}]
	\par Таким образом, алгоритм выглядит следующим образом:
	\begin{enumerate}
		\item для каждой буквы исходной строки найти индекс $i$ в алфавите;
		\item добавить в результирующую строку букву с индексом \begin{math} (i+key)\%len(alphabet)\end{math} в алфавите.
	\end{enumerate}

\section{XOR-шифр}
	\par $XOR$-шифрование -- это применение ключа через побитовое исключающее ИЛИ к исходному тексту.
	\par Представлена таблица истинности побитового исключающего ИЛИ:
	\begin{table}[h]
	\begin{center}
		\caption{\label{table:xor} Таблица истинности побитового исключающего ИЛИ}
		\begin{tabular}{|c| c| c|} 
			\hline
			X & Y & XOR(X, Y) \\
			\hline\hline
			0 & 0 & 0\\
			\hline
			0 & 1 & 1\\
			\hline
			1 & 0 & 1\\
			\hline
			1 & 1 & 0\\
			\hline
		\end{tabular}
	\end{center}
	\end{table}
	\par Таким образом, при выполнении исключающего ИЛИ всегда будет нулевое значени, если переменные имели одинаковые значения, иначе будет единица.
	\par Особенность $XOR$ в том, что одной и той же функцией можно как зашифровать данные, так и расшифровать их. Это простой метод шифрации данных, который может быть взломан достаточно быстро при наличии достаточно большого зашифрованного текста, или большого словаря паролей. Но тем не менее это уже можно применять для небольшой первоначально защиты данных [\ref{bib:3}].

	\section*{Вывод}
\par В качестве входных данных для конвеера с тремя линиями достаточно использовать размер очереди, обрабатываемой конвейером.

Конвейер может быть реализован следующим образом:
\begin{enumerate}
	\item [Линия 1.] Шифрование строки $XOR$-шифром с первым ключом;

	\item [Линия 2.] Шифрование строки шифром Цезаря;

	\item [Линия 3.] Шифрование строки $XOR$-шифром со вторым ключом;
\end{enumerate}

\newpage