\chapter{Технологический раздел}
\label{cha:technological}

    В данном разделе представлены средства, использованные в процессе разработки для реализации задачи, а также листинг кода программы. Кроме того показаны результаты тестирования.

    \section{Требования к ПО}
        Программное обеспечение должно удовлетворять следующим требованиям:
        \begin{itemize}
            \item программа получает на вход размер очереди;
            \item программа выдает журналирвоание работы конвейера;
            \item генератор строк выдает ненулевые строки, состоящие из строчных и заглавных букв;
            \item шифрование каждым методом выполняется с разными ключами.
        \end{itemize}
    \section{Средства реализации}
        Для реализации ПО был выбран компилируемый многопоточный язык программирования Golang, поскольку язык отличается легкой и быстрой сборкой программ и автоматическим управлением памяти.

    \section{Листинг программы}
        В приведенных ниже листингах представлены следующие реализации: 
        \begin{enumerate}
            \item паралельного конвейера (листинги \ref{lst:alg:Parallel}, \ref{lst:alg:Parallel1}, \ref{lst:alg:Parallel2});
            \item синхронного конвейера (листинги \ref{lst:alg:Sync}, \ref{lst:alg:Sync1});
            \item XOR-шифра (листинг \ref{lst:alg:Xor});
            \item шифра Цезаря (листинг \ref{lst:alg:Caesar}).
        \end{enumerate}
        \par \text{           }
        \begin{lstlisting}[language=Go, label=lst:alg:Parallel, caption=Реализация параллельного конвейера]
func Pipeline(count int, ch chan int) *Queue {

    first := make(chan *PipeTask, count)
    second := make(chan *PipeTask, count)
    third := make(chan *PipeTask, count)

    line := initQueue(count)
\end{lstlisting}
\par \text{           }
\par \text{           }
\par \text{           }
\par \text{           }
\begin{lstlisting}[language=Go, label=lst:alg:Parallel1, caption=Реализация параллельного конвейера]
    gen_string := func() {
        for {
            select {
            case pipe_task := <-first:
                pipe_task.generated = true

                len_s := rand.Intn(100)
                pipe_task.source = chiperalg.GenerateRune(len_s)
                pipe_task.start_generating = time.Now()
                pipe_task.dst = chiperalg.XorChiperFirst(pipe_task.source, len_s)

                pipe_task.end_generatig = time.Now()

                second <- pipe_task
            }
        }
    }

    get_table := func() {
        for {
            select {
            case pipe_task := <-second:
                pipe_task.skip_table_made = true

                pipe_task.start_table = time.Now()

                pipe_task.dst = chiperalg.CaesarChiper(pipe_task.dst, len(pipe_task.dst))
                pipe_task.end_table = time.Now()

                third <- pipe_task
            }
        }
    }
    match := func() {
        for {
            select {
            case pipe_task := <-third:
                pipe_task.pattern_mached = true

                pipe_task.start_match = time.Now()
                pipe_task.dst = chiperalg.XorChiperSecond(pipe_task.dst, len(pipe_task.dst))
                pipe_task.end_match = time.Now()
\end{lstlisting}

\begin{lstlisting}[language=Go, label=lst:alg:Parallel2, caption=Реализация параллельного конвейера]

                line.enqueue(pipe_task)
                if pipe_task.num == count-1 {
                    ch <- 0
                }
            }
        }
    }

    go gen_string()
    go get_table()
    go match()

    for i := 0; i < count; i++ {
        pipe_task := new(PipeTask)
        pipe_task.num = i + 1
        first <- pipe_task
    }
    return line
}
\end{lstlisting}
\par \text{  }
\begin{lstlisting}[language=C++, label=lst:alg:Sync, caption=Реализация синхронного конвейера]
func chiper_xor_first(task *PipeTask) *PipeTask {
    task.generated = true

    len_s := rand.Intn(100)
    task.source = chiperalg.GenerateRune(len_s)
    task.start_generating = time.Now()
    task.dst = chiperalg.XorChiperFirst(task.source, len_s)

    task.end_generatig = time.Now()

    return task
}

func chiper_caesar(task *PipeTask) *PipeTask {
    task.skip_table_made = true

    task.start_table = time.Now()
\end{lstlisting}
\par \text{  }
\par \text{  }
\par \text{  }
\begin{lstlisting}[language=C++, label=lst:alg:Sync1, caption=Реализация синхронного конвейера]
    task.dst = chiperalg.CaesarChiper(task.dst, len(task.dst))
    task.end_table = time.Now()

    return task
}

func chiper_xor_second(task *PipeTask) *PipeTask {
    task.pattern_mached = true

    task.start_match = time.Now()
    task.dst = chiperalg.XorChiperSecond(task.dst, len(task.dst))
    task.end_match = time.Now()

    return task
}

func Sync(count int) *Queue {

    line_first := initQueue(count)
    line_second := initQueue(count)
    line_third := initQueue(count)

    for i := 0; i < count; i++ {
        pipe_task := new(PipeTask)
        pipe_task = chiper_xor_first(pipe_task)
        line_first.enqueue(pipe_task)
        if len(line_first.queue) != 0 {
            pipe_task = chiper_caesar(line_first.dequeue())
            line_second.enqueue(pipe_task)
            if len(line_second.queue) != 0 {
                pipe_task = chiper_xor_second(line_second.dequeue())
                line_third.enqueue(pipe_task)
            }
        }
    }
    return line_third
}
        \end{lstlisting}
\par \text{ }
\par \text{ }
\par \text{ }
\par \text{ }
\par \text{ }
\par \text{  }
\begin{lstlisting}[language=C++, label=lst:alg:Xor, caption=Реализация алгоритма XOR шифрования]
func XorChiperFirst(src []rune, size int) []rune {

    var smeshenie = 8
    var itog = make([]rune, size)
    for i := range src {
        itog[i] = rune(int(src[i]) ^ smeshenie)
    }
    return itog
}

func XorChiperSecond(src []rune, size int) []rune {

    var smeshenie = 3
    var itog = make([]rune, size)
    for i := range src {
        itog[i] = rune(int(src[i]) ^ smeshenie)
    }
    return itog
}
\end{lstlisting}
\par \text{ }
\begin{lstlisting}[language=C++, label=lst:alg:Caesar, caption=Реализация алгоритма шифра Цезаря]
func CaesarChiper(src []rune, size int) []rune {
    var letters = []rune("abcdefghijklmnopqrstuvwxyzABCDEFGHIJKLMNOPQRSTUVWXYZ")
    var smeshenie = 6
    var itog = make([]rune, size)
    for i := range src {
        var mesto = search(letters, src[i])
        var newmesto = (mesto + smeshenie) % len(letters)
        if search(letters, src[i]) > -1 {
            itog[i] = letters[newmesto]
        } else {
            itog[i] = src[i]
        }
    }
    return itog
}
        \end{lstlisting}

        
    \section{Тестирование ПО}

        Тестирование ПО проводилось написанием журналирования для вывода
        содержания заявок очереди
        \par \text{  }
\begin{lstlisting}[language=C++, label=lst:testing, caption=Тестирование ПО]
func Log(qu *Queue) {
    fmt.Println("")
    line := qu.queue
    for i := range line {
        if line[i] != nil {
            fmt.Printf("%3v & %10v & %10v & %10v & %10v &  \n", i, string(line[i].source), string(line[i].dst1), string(line[i].dst2), string(line[i].dst3))
        }
    }
}
        \end{lstlisting}

        В таблице \ref{table:testing} отображёны результаты на очереди размером в 5 элементов.

        \begin{table}[h!]
            \caption{Тесты проверки корректности программы}
            \centering
\begin{tabular}{|l|l|l|l|l|}
\hline
№ & Source  & XOR key-1                  & Caesar                     & XOR key-2  \\ \hline
1 & k       & c                          & i                          & j          \\ \hline
2 & sBtgPSb & \{J|oX{[}j                 & \{P|ud{[}p                 & xS\_vgXs   \\ \hline
3 & cw      & k\_                        & q\_                        & r|         \\ \hline
4 & XDpVR   & PLx\textasciicircum{}Z     & VRD\textasciicircum{}f     & UQG{]}e    \\ \hline
5 & LMwwVtm & DE\_\_\textasciicircum{}|e & JK\_\_\textasciicircum{}|k & IH||{]}\_h \\ \hline
\end{tabular}
            \label{table:testing}
        \end{table}

	\section*{Вывод}
	
	Было написано и протестировано программное обеспечение для решения поставленной задачи.
    	
\newpage