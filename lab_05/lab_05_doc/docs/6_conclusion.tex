\section*{\large Заключение}
\addcontentsline{toc}{chapter}{Заключение}
    \par В рамках данной лабораторной работы была достигнута её цель: изучены конвеерные вычисления. Также выполнены следующие задачи:
    \begin{itemize}
    	\item была изучена организация конвеерной обработки данных;
		\item были изучены и реализованы шифр Цезаря и $xor$-шифр;
		\item были проведены замеры процессорного времени работы для реализованных алгоритмов.
	\end{itemize}
	\par Вычислительный конвейер был реализован на трех потоках и на одном потоке. Алгоритмы шифрования Цезаря и XOR-шифра были применены для обработки конвейерными линиями. Эксперимент показал, что распараллеливание вычислительного конвейера приводит к выигрышу в 6\%. Выигрыш происходит исключительно за счет обеспечения меньшего простоя очереди и ситуаций, когда простоя нет вовсе.
\newpage
