% % Список литературы при помощи BibTeX
% Юзать так:
%
% pdflatex report
% bibtex report
% pdflatex report

\bibliographystyle{gost780u}

\section*{\large Список источников}
\addcontentsline{toc}{chapter}{Список источников}
\begin{enumerate}
	\item Pipelining: Basic Concepts and Approaches // [Электронный ресурс]. Режим доступа: \url{https://www.ijser.org/researchpaper/Pipelining-Basic-Concepts-And-Approaches.pdf} дата обращения: 10.12.2022); \label{bib:1}
	\item Шифр Цезаря или как просто зашифровать текст. // [Электронный ресурс]. Режим доступа: \url{https://habr.com/ru/post/534058/} (дата обращения: 10.12.2022); \label{bib:2}
	\item Шифрование методом XOR. // [Электронный ресурс]. Режим доступа: 
	\url{https://evileg.com/ru/post/271/} (дата обращения: 10.12.2022). \label{bib:3}
	\item Donald Knuth. “The Art of Computer Programming. Volume 1: Fundamental Algorithms, Third Edition.” в: Addison-Wesley, 1997. гл. Stacks, Queues, and Dequeues, с. 238—243.\label{bib:4}
	\item Документация по операционной системе macOS Ventura. [Электронный ресурс]. Режим доступа: \url{https://www.apple.com/macos/ventura/} (дата обращения: 10.12.2022). \label{bib:11}
	\item Документация по процессору Apple M1. [Электронный ресурс]. Режим доступа: \url{https://www.apple.com/ru/newsroom/2020/11/apple-unleashes-m1/} (дата обращения: 10.12.2022). \label{bib:12}
	\item Rune in Golang. [Электронный ресурс]. Режим доступа: \url{https://www.geeksforgeeks.org/rune-in-golang/} (дата обращения: 10.12.2022). \label{bib:7}
\end{enumerate}