\section*{\large Введение}
\addcontentsline{toc}{chapter}{Введение}

\par Время работы является одной из основных характеристик, которые влияют на оценку того или иного алгоритма. У большинства алгоритмов можно найти модифицированные аналоги, которые стараются улучшить эту характеристику. Сделать алгоритм более быстрым может не только его модификация, но и такой способ как организация вычислительного конвеера.
\par Метод вычислительного конвеера предполагает декомпозицию задачи на подзадачи таким образом, что результат работы одной подзазачи является входными данными для следующей подзадачи. Таком образом реализуются конвеерные вычисления [\ref{bib:1}].
\par В представленной работе исследуется реализация вычислительного конвеера, который используется в качестве декомпозированных подзадач алгоритмы шифрования строк: шифр Цезаря и $xor$-шифр.

	Цель лабораторной работы:\\
	изучить и реализовать конвеерные вычисления.

	Задачи лабораторной работы:
	\begin{enumerate}
		\item изучить организацию конвеерной обработки данных;
		\item изучить шифр Цезаря и $xor$-шифр;
		\item реализовать:
		\begin{enumerate}
			\item последовательную конвеерную обработку данных;
			\item параллельную конвеерную обработку данных.
		\end{enumerate}
		\item провести замеры процессорного времени работы для реализованных алгоритмов.
	\end{enumerate}
\newpage