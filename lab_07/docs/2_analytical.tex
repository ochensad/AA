\chapter{Аналитический раздел}
\label{cha:analytical}

	\section{Нечеткие переменные}

	\par \textbf{Лингвистической переменной} являются слова или предложения ествственного языка, которые описываются нечеткими значениями.

	\par В случаях, когда требуется формализовать разделение диапозона значений данной величины на некоторые категории следует выделить набор категорий и определить границы диапозонов значений в пределах каждой категории. Для этого требуется агрегировать набор значений и интерпретировать набор мнений экспертов касательно соотношения категории и конкретных значений описываемой величины [\ref{bib:1}].
	\par При построении функции принадлежности значений категориям используются два множеста:
	\begin{enumerate}
		\item множество термов $\tau$, которое описывает лингвистических терм $\tau_i' = (t_1, t_2, ..., t_m)$, \\$i \in [1,m]$ на универсальном множестве \begin{math} X = (x_1, x_2, ... ,x_n)\end{math}
		\item собственно само универсальное множество \begin{math} X = (x_1, x_2, ... ,x_n)\end{math}
	\end{enumerate}
	\par \textbf{Термом} являются слова или предложения ествственного языка, которые описываются нечеткими значениями.
	\par \textbf{Универсальное множество} -- рассматриваемое множество значений.

	\begin{equation}
	\tau_i' = \left( \frac{\mu_{\tau_i}(x_1)}{x_1}, \frac{\mu_{\tau_i}(x_2)}{x_2}, ..., \frac{\mu_{\tau_i}(x_n)}{x_n}\right)
	\end{equation}

	\par Требуется определить для любых \begin{math}i \in [1,m], j \in [1,n]\end{math} степени функций принадлежности элементов множества $X$ к элементам из множества \begin{math}\tau_i'\end{math}, т.е. \begin{math} \mu_{\tau_i}(x_j) \end{math}.
	\par Для решения данной задачи будет использовать метод на основе статитической обработки мнений экспертов (респондентов).

	\section{Анкетирование}

	\par В данном методе усредняют знания коллектива специалистов относительно распределения экспертов по универсальному множеству. Анкета будет иметь следующий вид:

	\begin{table}[!h]
	\centering
    \caption{Анкета для $k$ экспертов}
\begin{tabular}{|l|l|l|l|l|l|}
\hline
                   &     & $x_1$   & $x_2$   & ... & $x_n$   \\ \hline
\multirow{4}{*}{$\tau_1'$} & $t_1$  & $a_{11}^1$ & $a_{21}^1$ & ... & $a_{n1}^1$ \\ \cline{2-6} 
                   & $t_2$  & $a_{12}^1$ & $a_{22}^1$ & ... & $a_{2n}^1$ \\ \cline{2-6} 
                   & ... & ...  & ...  & ... & ...  \\ \cline{2-6} 
                   & $t_m$  & $a_{1m}^1$ & $a_{2m}^1$ & ... & $a_{nm}^1$ \\ \hline
\multirow{4}{*}{$\tau_2'$} & $t_1$  & $a_{11}^2$ & $a_{21}^2$ & ... & $a_{n1}^2$ \\ \cline{2-6} 
                   & $t_2$  & $a_{12}^2$ & $a_{22}^2$ & ... & $a_{2n}^2$ \\ \cline{2-6} 
                   & ... & ...  & ...  & ... & ...  \\ \cline{2-6} 
                   & $t_m$  & $a_{1m}^2$ & $a_{2m}^2$ & ... & $a_{nm}^2$ \\ \hline
...                & ... & ...  & ...  & ... & ...  \\ \hline
\multirow{4}{*}{$\tau_k'$} & $t_1$  & $a_{11}^k$ & $a_{21}^k$ & ... & $a_{n1}^k$ \\ \cline{2-6} 
                   & $t_2$  & $a_{12}^k$ & $a_{22}^k$ & ... & $a_{n2}^k$ \\ \cline{2-6} 
                   & ... & ...  & ...  & ... & ...  \\ \cline{2-6} 
                   & $t_m$  & $a_{1m}^k$ & $a_{2m}^k$ & ... & $a_{nm}^k$ \\ \hline
\end{tabular}
\label{tab:anket}
\end{table}
	где $k$ -- количество экспертов, $a_{nm}^k \in [0,1]$ -- результат бинарной экспертной оценки $k$-м экспертов у элемента $x_n$ свойств нечеткого множества $\tau_i', j \in [1,n], i \in [1,m],k \in [1,p]$.
	\par По результатам анкетирования рассчитываются степени функции принадлежностью нечетному множеству:
	\begin{equation}
	\tau_i': \mu_{\tau_i'}(x_n) = \frac{1}{k} \sum^{k}{a_{nm}^k}
	\end{equation}\label{formula:mu}

	\section{Формализация объекта}
	\par В качестве лингвистической переменной в данной работе будет рассмотрен размер отчета по Анализу алгоритмов. Она принимает нечеткие значения:
	\begin{enumerate}
		\item «очень маленький»;
		\item «маленький»;
		\item «небольшой»;
		\item «средний»;
		\item «большой»;
		\item «очень большой».
	\end{enumerate}
	\par Эти нечеткие значения образуют множество термов.
	\par Данное множество термов будет рассмотрено на универсальном множестве \begin{math} X = [0,50]\end{math}.
	\section{Словарь как структура данных}

Словарь (или «\textit{ассоциативный массив}») \ref{bib:2} - абстрактный тип данных (интерфейс к хранилищу данных), позволяющий хранить пары вида «(ключ, значение)» и поддерживающий операции добавления пары, а также поиска и удаления пары по ключу:
\begin{itemize}
	\item \texttt{INSERT(k, v)};
	\item \texttt{FIND(k)};
	\item \texttt{REMOVE(k)}.
\end{itemize}

В паре \texttt{(k, v)}: \texttt{v} называется значением, ассоциированным с ключом \texttt{k}. Где \texttt{k} — это ключ, a \texttt{v} — значение. Семантика и названия вышеупомянутых операций в разных реализациях ассоциативного массива могут отличаться.

Операция \texttt{ПОИСК(k)} возвращает значение, ассоциированное с заданным ключом, или некоторый специальный объект \texttt{НЕ\_НАЙДЕНО}, означающий, что значения, ассоциированного с заданным ключом, нет. Две другие операции ничего не возвращают (за исключением, возможно, информации о том, успешно ли была выполнена данная операция).

Ассоциативный массив с точки зрения интерфейса удобно рассматривать как обычный массив, в котором в качестве индексов можно использовать не только целые числа, но и значения других типов — например, строки.

В данной лабораторной работе в качестве ключа будет использоваться строка: название отчета, а в качестве значения -- целое число: количество страниц в данном отчете.

\section{Алгоритм полного перебора}
Алгоритмом полного перебора \ref{bib:3} называют метод решения задачи, при котором по очереди рассматриваются все возможные варианты. В случае реализации алгоритма в рамках данной работы будут последовательно перебираться ключи словаря до тех пор, пока не будет найден нужный. 

Трудоёмкость алгоритма зависит от того, присутствует ли искомый ключ в словаре, и, если присутствует -- насколько он далеко от начала массива ключей.

Пусть на старте алгоритм затрагивает $k_{0}$ операций, а при сравнении $k_{1}$ операций. 

Пусть алгоритм нашёл элемент на первом сравнении (лучший случай), тогда будет затрачено $k_0 + k_1$ операций, на втором - $k_0 + 2 \cdot k_1$, на последнем (худший случай) - $k_0 + N \cdot k_1$. Если ключа нет в массиве ключей, то мы сможем понять это, только перебрав все ключи, таким образом трудоёмкость такого случая равно трудоёмкости случая с ключом на последней позиции. Средняя трудоёмкость может быть рассчитана как математическое ожидание по формуле (\ref{for:brute}), где $\Omega$ -- множество всех возможных случаев.

\begin{equation}
\label{for:brute}
\begin{aligned}
\sum\limits_{i \in \Omega} p_i \cdot f_i = & (k_0 + k_1) \cdot \frac{1}{N + 1} + (k_0 + 2 \cdot k_1) \cdot \frac{1}{N+1} +\\& + (k_0 + 3 \cdot k_1) \cdot \frac{1}{N + 1} + (k_0 + Nk_1)\frac{1}{N + 1} + (k_0 + N \cdot k_1) \cdot \frac{1}{N + 1} =\\& = k_0\frac{N+1}{N+1}+k_1+\frac{1 + 2 + \cdots + N + N}{N + 1} = \\& = k_0 + k_1 \cdot \left(\frac{N}{N + 1} + \frac{N}{2}\right) = k_0 + k_1 \cdot \left(1 + \frac{N}{2} - \frac{1}{N + 1}\right)
\end{aligned}
\end{equation}


	\section*{Вывод}
\par Программное обеспечение, решающее поставленную задачу, может работать следующим образом. На вход алгоритму подается словарь, содержащий данные об отчетах и их размерах, а также запрос к этому словарю. Программа возвращает массив элементов, которые удовлетворяют данному запросу. 


\newpage