\section*{\large Заключение}
\addcontentsline{toc}{chapter}{Заключение}
    \par В рамках данной лабораторной работы была достигнута её цель: получен навык поиска по словарю при ограничении на значение признака, заданном при помощи лингвистической переменной. Также выполнены следующие задачи:
    \begin{itemize}
    	\item формализован объект и его признак;
		\item составлена анкета для её заполнения респондентами;
		\item проведено анкетирование респондентов;
		\item построена функция принадлежности термам числовых значений признака, описываемого лингвистической переменной, на основе статистической обработки мнений респондентов, выступающих в роли экспертов;
		\item описаны 4 типовых вопросов на русском языке, имеющих целью запрос на поиск в словаре;
		\item описан алгоритм поиска полным перебором в словаре объектов, удовлетворяющих ограничению, заданному в вопросе на ограниченном естественном языке;
		\item описана структуру данных словаря, хранящего наименования объектов согласно варианту и числовое значение признака объекта;
		\item реализован алгоритм поиска полным перебором в словаре;
		\item приведены примеры запросов пользователя и сформированный реализацией алгоритм поиска выборки объектов из словаря, используя составленные респондентами вопросы.
	\end{itemize}
	\par Предложенный в лабораторной работе алгоритм ограничено применим только в рамках поставленной задачи. Имеется возможность корректировать термы под определенную задачу, но только для универсального множества в виде целых чисел. Ограничения данного алгоритма заключаюся в том, что он не приспособлен для запросов не включающих установленных термов.
\newpage
