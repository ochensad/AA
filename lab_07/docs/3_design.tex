\chapter{ Констукторский раздел}
\label{cha:design}
    В данном разделе будет составлена анкета для ее заполнения респондентами, описаны типовые запросы; будет рассмотрена схема алгоритма и требования к функциональности ПО.
    
    \section{Структура анкеты}
    \par В таблице \ref{tab:anket1} представлена анкета для заполнения респондентами. Респондент должен ответить утверждением или нет для каждого из значений данной анкеты. 
    \begin{table}[!h]
    \centering
    \caption{Анкета для заполнения респондентами}
\begin{tabular}{|l|l|l|l|l|l|l|l|l|l|l|l|}
\hline
                & 0 & 5 & 10 & 15 & 20 & 25 & 30 & 35 & 40 & 45 & 50 \\ \hline
очень маленький &   &   &    &    &    &    &    &    &    &    &    \\ \hline
маленький       &   &   &    &    &    &    &    &    &    &    &    \\ \hline
небольшой       &   &   &    &    &    &    &    &    &    &    &    \\ \hline
средний         &   &   &    &    &    &    &    &    &    &    &    \\ \hline
большой         &   &   &    &    &    &    &    &    &    &    &    \\ \hline
очень большой   &   &   &    &    &    &    &    &    &    &    &    \\ \hline
\end{tabular}
\label{tab:anket1}
\end{table}
    \section{Типовые запросы}
    \par Таким образом типовые запросы к словарю выглядят так:
    \begin{itemize}
        \item «Вывести все небольшие отчеты»;
        \item «Найти большие и очень большие отчеты»;
        \item «Маленькие отчеты»;
        \item «Показать средние и небольшие отчеты».
    \end{itemize}
    \par Из типовых запросов видно, что запросы регистронезависимы и могут использовать более одного терма.


    \section{Описание используемых типов данных}
    \par При реализации алгоритмов будут использованы следующие типы данных:
    \begin{itemize}
        \item словарь –- встроенный тип dict [\ref{bib:4}] в Python[\ref{bib:5}] будет использован в качестве словаря;
        \item массив с результатами запроса –- встроенный тип list [\ref{bib:6}] в Python.
    \end{itemize}

    \section{Структура ПО}
    \par В данном ПО буде реализован метод структурного программирования. Взаимодействие с пользователем будет через консоль, будет дана возможность ввода запроса для поиска значений в словаре. Для работы будут разработаны следующие процедуры:

    \begin{itemize}
        \item главная процедура – является точкой входа в программу, входных данных нет, выходных данных нет;
        \item процедура обработки введенного запроса -- анализирует строку с запросом и вычленяет из нее разделители и термы для поиска;
        \item процедура поиска -- исходя из полученных массивов разделителей и термов возвращает массив результатов;
        \item процедура печати -- получает массив результатов и построчно выводит элементы массива на экран.
    \end{itemize}

    \section{Схема алгоритма}
    \par На рисунке \ref{alg:1} представлена схема алгоритма поиска в словаре полным перебором.

    \begin{figure}[h!]
            \centering
            \includegraphics[scale=0.9]{img/alg.pdf}
            \caption{Функция принадлежности для лингвистической переменной «размер отчета по Анализу алгоритмов»}
            \label{alg:1}
        \end{figure}
        \newpage

	\section*{Вывод}
    \par Были разработаны схемы алгоритмов, необходимых для решения задачи. Получено достаточно теоретической информации для написания программного обеспечения.
\newpage