\section*{\large Введение}
\addcontentsline{toc}{chapter}{Введение}

\par Словарь, как тип данных, применяется везде, где есть связь «ключ -- значение» или «объект -- данные»: поиск налогов по ИНН и другое. Поиск - основная задача при использовании словаря. Но также важно, правильно задать вопрос, чтобы поисковая система в словаре могла выдать однозначный результат.

	Цель лабораторной работы:\\
	получить навык поиска по словарю при ограничении на значение признака, заданном при помощи лингвистической переменной.

	Задачи лабораторной работы:
	\begin{enumerate}
		\item формализовать объект и его признак;
		\item составить анкету для её заполнения респондентом;
		\item провести анкетирование респондентов;
		\item построить функцию принадлежности термам числовых значений признака, описываемого лингвистической переменной, на основе статистической обработки мнений респондентов, выступающих в роли экспертов;
		\item описать 3-5 типовых вопросов на русском языке, имеющих целью запрос на поиск в словаре;
		\item описать алгоритм поиска в словаре объектов, удовлетворяющих ограничению, заданному в вопросе на ограниченном естественном языке;
		\item описать структуру данных словаря, хранящего наименования объектов согласно варианту и числовое значение признака объекта;
		\item реализовать алгоритм поиска в словаре;
		\item привести примеры запросов пользователя и сформированный реализацией алгоритм поиска выборки объектов из словаря, используя составленные респондентами вопросы;
		\item дать заключение о применимости предложенного алгоритма и о его ограничениях.
	\end{enumerate}

\newpage