\section*{\large Введение}
\addcontentsline{toc}{chapter}{Введение}

\par Алгоритм сортировки – это алгоритм для упорядочивания элементов в списке. Входом является последовательность из n элементов: \begin{math}a_1, a_2, ..., a_n\end{math}. Результатом работы алгоритма сортировки является перестановка исходной последовательности \begin{math}a_1', a_2', ..., a_n'\end{math}, такая что
\begin{math}a_1' \leq a_2'\leq ... \leq a_n'\end{math}, где \begin{math}\leq\end{math} – отношение порядка на множестве элементов списка. Поля, служащие критерием порядка, называются ключом сортировки. На практике в качестве ключа часто выступает число, а в остальных полях хранятся какие-либо данные, никак не влияющие на работу алгоритма.
	
	В данной лабораторной работе рассмотриваются алгоритмы:

	\begin{enumerate}
		\item сортировка плавная;
		\item сортировка расчёской;
		\item сортировка слиянием.
	\end{enumerate}

	Цель лабораторной работы:\\
	изучить трудоемкости алгоритмов сортировки и их реализация.

	Задачи лабораторной работы:
	\begin{enumerate}
		\item выбрать инструменты для замера процессорного времени выполнения реализации алгоритмов;
		\item изучить алгоритмы сортировки расчёской, слияением и плавной сортировки;
		\item реализовать:
		\begin{enumerate}
			\item плавную сортировку;
			\item сортировку расчёской;
			\item сортировку слиянием.
		\end{enumerate}
		\item дать оценку трудоёмкости в лучшем, произвольном и худшем случае;
		\item провести замеры процессорного времени работы для лучшего, худшего и произвольного случая.
	\end{enumerate}
\newpage