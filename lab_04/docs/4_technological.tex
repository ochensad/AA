\chapter{Технологический раздел}
\label{cha:technological}

    В данном разделе представлены средства, использованные в процессе разработки для реализации задачи, а также листинг кода программы. Кроме того показаны результаты тестирования и анализа затрачиваемой памяти.

    \section{Средства реализации}
        Для реализации поставленной задачи был использован язык C++, так как имеется большой опыт работы с ним. Для измерения процессорного времени была использована ассемблерная вставка. 

    \section{Сведения о модулях программы}
    \par Программа состоит из:
    \begin{itemize}
        \item main.cpp -- главный файл программы, в нем располагается точка входа;
        \item Alg.cpp -- содержит последовательный и параллельный нечёткие алгоритмы k-means;
        \item time.cpp -- содержит ассемблерную вставку для замера времени.
    \end{itemize}

    \section{Листинг программы}
        В приведенных ниже листингах представлены следующие реализации: 
        \begin{enumerate}
            \item последовательный нечёткий алгоритм k-means(листинг \ref{lst:alg:Fuzzy};
            \item параллельный нечёткий алгоритм k-means (листинг \ref{lst:alg:FuzzyPar}).
        \end{enumerate}
        \par \text{           }
        \begin{lstlisting}[language=C++, label=lst:alg:Fuzzy, caption=Последовательный нечёткий алгоритм k-means]
vector<vector<double>> FuzzyCMeans (vector<vector<double>> X, int c, int m, double eps)
{
    int n = X.size();
    int r = X[0].size();
    vector<vector<double>> centers = generateZeroFill(c, r);
    vector<vector<double>> Wprev = generateRandFill(n, c); 
    vector<vector<double>> W = Wprev;
    while (true){
        for (int i=0; i<c; i++)
        {
            vector<double> num(X[0].size());
            double den = 0.0;
            for (int j=0; j<n; j++)
            {
\end{lstlisting}
\par \text{  }
\begin{lstlisting}[language=C++, label=lst:alg:Fuzzy_1, caption=Последовательный нечёткий алгоритм k-means]

                num = SumVecVec(num, MulNumVec(pow(W[j][i], m),X[j]));
                den += pow(W[j][i], m);
            }
            centers[i] = DevVecNum(num, den);
        }
        double max_diff = 0.0;
        for (int i=0; i<n; i++)
            for (int j=0; j<c; j++)
            {
                double total = 0.0;
                for (int k=0; k<c; k++)
                {
                    total += pow(distance(X[i], centers[j]) / distance(X[i], centers[k]), 2/(m-1));
                }
                W[i][j] = 1.0/total;
                max_diff = max(fabs(W[i][j]-Wprev[i][j]), max_diff);
                Wprev[i][j] = W[i][j];
            }
        if (max_diff < eps)
            break;
    }
    return W;
}
        \end{lstlisting}
\begin{lstlisting}[language=C++, label=lst:alg:FuzzyPar, caption=Параллельный нечёткий алгоритм k-means]
vector<vector<double>> FuzzyCMeansParallel(vector<vector<double>> X, int c, int m, double eps, int threads)
{
    int n = X.size();
    int r = X[0].size();
    vector<vector<double>> centers = generateZeroFill(c, r);
    vector<vector<double>> Wprev = generateRandFill(n, c); 
    vector<vector<double>> W = Wprev;
    while (true){
        // обновляем центры кластеров
        for (int i=0; i<c; i++)
        {
            vector<double> num(X[0].size());
            double den = 0.0;
            for (int j=0; j<n; j++)
            {
\end{lstlisting}
\par \text{ }
\begin{lstlisting}[language=C++, label=lst:alg:FuzzyPar_1, caption=Параллельный нечёткий алгоритм k-means]
                num = SumVecVec(num, MulNumVec(pow(W[j][i], m),X[j]));
                den += pow(W[j][i], m);
            }
            centers[i] = DevVecNum(num, den);
        }
           //обновляем вероятности W и считаем расстояние между матрицами W и Wprev
        vector<double> maxs(threads);
        int count = n / threads;
        int k = 0;
        vector<thread> threads_arr;
        for(int i = 0; i < threads; i++)
        {
            for(int j = 1; j < count+1; j++, k++)
                ;
            if (i == threads - 1 && k != n)
            {
                k++;
            }
            threads_arr.push_back(std::thread(ParallelMatrix, std::ref(W), std::ref(Wprev), std::ref(maxs[i]),i*count,k,c, X, centers, m));
        }
        double max_diff = 0.0;
        for(int i = 0; i < threads; i++)
        {
            threads_arr[i].join();
            max_diff = max(maxs[i], max_diff);
            
        }
        if (max_diff < eps)
            break;
    }
    return W;
}
        \end{lstlisting}

        
    \section{Тестирование}
        В таблице \ref{table:testing} отображён возможный набор тестов
        для тестирования методом чёрного ящика, в соответствующих столбцах таблицы представлены результаты тестирования.
        \begin{table}[h!]
            \caption{Тесты проверки корректности программы}
            \centering
            \begin{tabular}{|c|c|c|c|}
            \hline
            № & Input & Fuzzy & FuzzyPar \\\hline
            1 & $\begin{bmatrix} 1 & 1 \\ 2 & 2 \end{bmatrix}$ & $\begin{bmatrix} 0 & nan \\ nan & 0 \end{bmatrix}$ & $\begin{bmatrix} 0 & nan \\ nan & 0 \end{bmatrix}$  \\ \hline
            2 & $\begin{bmatrix} 1.4 & 1.0 & 1.3 \\ 4.3 & 4.0 & 4.8 \\ 9.5 & 9.2 & 9.1 \end{bmatrix}$ & 
            $\begin{bmatrix} 0.0221843 & 0.977816 \\
0.208947 & 0.791053 \\
0.999001 & 0.000998518 \end{bmatrix}$  & $\begin{bmatrix} 0.0221843 & 0.977816 \\
0.208947 & 0.791053 \\
0.999001 & 0.000998518 \end{bmatrix}$  \\ \hline
            3 & $\begin{bmatrix} 1.4 & 1.4 \\ 2.5 & 2.4 \\ 10.3 & 10.4 \\ 6.3 & 6.4 \end{bmatrix}$ & $\begin{bmatrix} 0.00742315 & 0.992577 \\ 
0.0043424 & 0.995658 \\
0.968385 & 0.0316153 \\
0.754249 & 0.245751  \end{bmatrix}$ & $\begin{bmatrix} 0.00742315 & 0.992577 \\ 
0.0043424 & 0.995658 \\
0.968385 & 0.0316153 \\
0.754249 & 0.245751  \end{bmatrix}$ \\ \hline
            \end{tabular}
            \label{table:testing}
        \end{table}

	\section*{Вывод}
	
	
	Были разработаны и протестированы спроектированные алгоритмы и указаны средства реализации поставленной задачи.
    	
\newpage