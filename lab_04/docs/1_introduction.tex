\section*{\large Введение}
\addcontentsline{toc}{chapter}{Введение}

\par Многопоточность — способность центрального процессора (CPU) или одного ядра в многоядерном процессоре одновременно выполнять несколько процессов или потоков, соответствующим образом поддерживаемых операционной системой.
\par Этот подход отличается от многопроцессорности, так как многопоточность процессов и потоков совместно использует ресурсы одного или нескольких ядер: вычислительных блоков, кэш-памяти ЦПУ или буфера перевода с преобразованием (TLB).
\par В тех случаях, когда многопроцессорные системы включают в себя несколько полных блоков обработки, многопоточность направлена на максимизацию использования ресурсов одного ядра, используя параллелизм на уровне потоков, а также на уровне инструкций.
\par Поскольку эти два метода являются взаимодополняющими, их иногда объединяют в системах с несколькими многопоточными ЦП и в ЦП с несколькими многопоточными ядрами. Многопоточная парадигма стала более популярной с конца 1990-х годов, поскольку
усилия по дальнейшему использованию параллелизма на уровне инструкций застопорились.
\par Смысл многопоточности — квазимногозадачность на уровне одного исполняемого процесса. [\ref{bib:3}]
	
	В данной лабораторной работе рассмотриваются алгоритмы:

	\begin{enumerate}
		\item последовательный алгоритм k-means;
		\item параллельный алгоритм k-means.
	\end{enumerate}

	Цель лабораторной работы:\\
	изучить и реализовать параллельные вычисления.

	Задачи лабораторной работы:
	\begin{enumerate}
		\item изучить понятие параллельных вычислений;
		\item изучить нечеткий алгоритм k-means;
		\item реализовать:
		\begin{enumerate}
			\item последоательный алгоритм k-means;
			\item параллельный алгоритм k-means.
		\end{enumerate}
		\item провести замеры процессорного времени работы для реализованных алгоритмов.
	\end{enumerate}
\newpage