% % Список литературы при помощи BibTeX
% Юзать так:
%
% pdflatex report
% bibtex report
% pdflatex report

\bibliographystyle{gost780u}

\section*{\large Список источников}
\addcontentsline{toc}{chapter}{Список источников}
\begin{enumerate}
	\item Ассемблерные вставки в AVR-GCC. // [Электронный ресурс]. Режим доступа: \url{https://habr.com/ru/post/275937/} дата обращения: 20.10.2022); \label{bib:1}
	\item C/C++: как измерять процессорное время. // [Электронный ресурс]. Режим доступа: \url{https://habr.com/ru/post/282301/} (дата обращения: 20.10.2022); \label{bib:2}
	\item Многопоточность в C++. Основные понятия. // [Электронный ресурс]. Режим доступа: 
	\url{https://radioprog.ru/post/1402} (дата обращения: 20.10.2022). \label{bib:3}
	\item Автоматическая обработка текстов на естественном языке и компьютерная лингвистика : учеб. пособие / Большакова Е.И., Клышинский Э.С., Ландэ Д.В., Носков А.А., Пескова О.В., Ягунова Е.В. — М.: МИЭМ, 2011. — 272 с. \label{bib:4}
\end{enumerate}