\section*{\large Заключение}
\addcontentsline{toc}{chapter}{Заключение}
    \par В рамках данной лабораторной работы была достигнута её цель: изучены паралелльные вычисления. Также выполнены следующие задачи:
    \begin{itemize}
    	\item было изучено понятие параллельных вычислений;
		\item были реализованы обычный и параллельная реализации нечеткого алгоритма k-means;
		\item было произведено сравнение временных характеристик реализованных алгоритмов экспериментально.
	\end{itemize}
	\par Параллельные реализации алгоритмов выигрывают по скорости у обычной (однопоточной) реализации нечеткого алгоритма k-means. Наиболее эффективны данные алгоритмы при количестве потоков, совпадающем с количеством логических ядер компьютера. Так, например, на матрицах размером 512 на 512, удалось улучшить время выполнения нечеткого алгоритма k-means в 2.5 раза (в сравнении с однопоточной реализацией).
\newpage
