\chapter{Технологический раздел}
\label{cha:technological}

    В данном разделе представлены средства, использованные в процессе разработки для реализации задачи, а также листинг кода программы. Кроме того показаны результаты тестирования.

    \section{Требования к ПО}
        Программное обеспечение должно удовлетворять следующим требованиям:
        \begin{itemize}
            \item программа принимает на вход количество дней, матрицу смежности графа и регулируемые параметры;
            \item программа выдает массив кратчайших путей, выходящих из каждой вершины графа.
        \end{itemize}
    \section{Средства реализации}
        Для реализации ПО был выбран компилируемый многопоточный язык программирования Golang, поскольку язык отличается легкой и быстрой сборкой программ и автоматическим управлением памяти.

    \section{Листинг программы}
        В приведенных ниже листингах представлены следующие реализации: 
        \begin{enumerate}
            \item муравьиного алгоритма (листинги \ref{lst:alg:Parallel}, \ref{lst:alg:Parallel1}, \ref{lst:alg:Parallel2});
            \item алгоритма полного перебора (листинги \ref{lst:alg:Sync}).
        \end{enumerate}
        \par \text{           }
        \begin{lstlisting}[language=Go, label=lst:alg:Parallel, caption=Реализация муравьиного алгоритма]
func ACOWrapper(gr_matrix [][]int, iters int, conf_filename string) *[]int {
    shortest := make([]int, len(gr_matrix))

    colony := makeColony(gr_matrix, iters, conf_filename)
    if (colony == nil) {
        return nil
    }

    shortest_tours := make([][]int, len(gr_matrix))
    for i := range gr_matrix {
        shortest_tours[i] = make([]int, len(gr_matrix[i]))
    }

    for i := 0; i < colony.iters; i++ {
        for j := 0; j < len(colony.gr_matrix) - 5; j++ {
            ant := colony.placeAnt(j)
            ant.elit_ant = true;
\end{lstlisting}

\begin{lstlisting}[language=Go, label=lst:alg:Parallel1, caption=Реализация муравьиного алгоритма]
            ant.startTour()
            current := ant.tourLength()
            if (current < shortest[j] || shortest[j] == 0) && len(ant.memory_tour) ==  len(colony.gr_matrix){
                shortest[j] = current
                colony.shortest_tour = ant.memory_tour
                copy(shortest_tours[j], ant.memory_tour)
            }
        }
        for j := len(colony.gr_matrix) - 5; j < len(colony.gr_matrix); j++ {
            ant := colony.placeAnt(j)
            ant.elit_ant = false
            ant.startTour()
            current := ant.tourLength()
            if (current < shortest[j] || shortest[j] == 0) && len(ant.memory_tour) ==  len(colony.gr_matrix){
                shortest[j] = current
                colony.shortest_tour = ant.memory_tour
                copy(shortest_tours[j], ant.memory_tour)
            }
        }
    }
    return &shortest
}
\end{lstlisting}

\par \text{           }
\begin{lstlisting}[language=Go, label=lst:alg:Parallel2, caption=Реализация перемещений муравья]
func (ant *Ant) startTour() {
    next := -1

    for flag := true; flag; flag = (next != -1) {
        vis := ant.getVision()
        next = fortuneWheel(vis)
        if (next != -1) {
            ant.tourAnt(next)
            ant.updatePhTrace()
        }
    } 
}
\end{lstlisting}
\par \text{           }
\par \text{           }
\par \text{           }
\par \text{           }

\begin{lstlisting}[language=Go, label=lst:alg:Parallel3, caption=Реализация функций для рассчета перемещений муравья]
func (ant *Ant) getVision() []float64 {
    var (
        vis = make([]float64, 0)
        denom = 0.0
    )
    ant.memory_tour = append(ant.memory_tour,ant.position)
    for i, l := range ant.tour[ant.position] {
        if l != 0 {
            term := math.Pow((1.0 / float64(l)), ant.colony.conf.TRACE_WEIGHT) * +
                math.Pow(ant.colony.phero_matrix[ant.position][i], ant.colony.conf.TOUR_VISIB)
            denom += term
            vis = append(vis, term)
        } else {
            vis = append(vis, 0)
        }
    }

    for i := 0; i < len(vis); i++ {
        vis[i] /= denom
    }

    return vis
}

func fortuneWheel(vis []float64) int {
    var (
        cumul_sum = make([]float64, len(vis))
        coin = rand.New(rand.NewSource(time.Now().UnixNano())).Float64()
        chosen = -1
    )

    for i := 0; i < len(vis); i++ {
        for j := i; j < len(vis); j++ {
            cumul_sum[i] += vis[j]
        }
    }
    for i := 0; i < len(cumul_sum); i++ {
        if i == len(cumul_sum) - 1 {
            if 0.0 <= coin && coin <= cumul_sum[i] {
                chosen = i
            }
        } else {
            if cumul_sum[i + 1] < coin && coin <= cumul_sum[i] {
                    chosen = i
\end{lstlisting}

\begin{lstlisting}[language=Go, label=lst:alg:Parallel4, caption=Реализация функций для рассчета перемещений муравья]
            }
        }
    }

    return chosen
}
\end{lstlisting}
\par \text{  }

\begin{lstlisting}[language=Go, label=lst:alg:Parallel5, caption=Реализация функции обновления феромона]
func (ant *Ant) updatePhTrace() {
    delta := 0.0
    for i := 0; i < len(ant.colony.phero_matrix); i++ {
        for j, phero := range ant.colony.phero_matrix[i] {
            if ant.colony.gr_matrix[i][j] != 0 {
                if ant.memory[i][j] {
                    delta = ant.colony.conf.Q / float64(ant.colony.gr_matrix[i][j])
                } else {
                    delta = 0
                }
                ant.colony.phero_matrix[i][j] = (1 - ant.colony.conf.EVAP_RATE) * (float64(phero) + delta)
                ant.colony.phero_matrix[j][i] = (1 - ant.colony.conf.EVAP_RATE) * (float64(phero) + delta)
            }
            if ant.colony.phero_matrix[i][j] <= 0 {
                ant.colony.phero_matrix[i][j] = ant.colony.conf.PHERO_EPS
                ant.colony.phero_matrix[j][i] = ant.colony.conf.PHERO_EPS
            }
        }
    }
    if ant.elit_ant == true && len(ant.colony.shortest_tour) != 0 {
        L := 0
        for i:= 0; i < len(ant.colony.shortest_tour) - 1; i++ {
            L += ant.colony.gr_matrix[ant.colony.shortest_tour[i]]
            [ant.colony.shortest_tour[i+1]]
        }
        for i:= 0; i < len(ant.colony.shortest_tour) - 1; i++ {
            ant.colony.phero_matrix[ant.colony.shortest_tour[i]]
            [ant.colony.shortest_tour[i + 1]] += ant.colony.conf.Q / float64(L)
        }
        ant.colony.phero_matrix[ant.colony.shortest_tour[len(ant.colony.shortest_tour) - 1]][ant.colony.shortest_tour[0]] += ant.colony.conf.Q / float64(L)
    }
}
\end{lstlisting}

\begin{lstlisting}[language=C++, label=lst:alg:Sync, caption=Реализация алгоритма полного перебора]
func BruteForce(gr_matrix [][]int) []int {
    var (
        roots = make([]int, 0)
        graph_len = len(gr_matrix)
        min_path = -1
    )

    for source := 0; source < graph_len; source++ {
        cur_roots := make([]int, 0)
        for i := 0; i < graph_len; i++ {
            if i != source {
                cur_roots = append(cur_roots, i)
            }
        } 
        next_permutation := permutations(cur_roots)
        for _, perm := range next_permutation {
            curr_cost := 0
            k := source
            flag := true

            for _, j := range perm {
                curr_cost += gr_matrix[k][j]
                if gr_matrix[k][j] == 0{
                    flag = false
                }
                k = j
            }

            if gr_matrix[k][source] == 0 {
                flag = false
            }
            curr_cost += gr_matrix[k][source]

            if (curr_cost < min_path || min_path == -1) && curr_cost != 0 && flag == true{
                min_path = curr_cost
            }
        }

        roots = append(roots, min_path)
        min_path = -1
    }
    return roots
}
\end{lstlisting}
\par \text{  }


    \subsection*{Автоматическая параметризация}

    Скрипт автоматической перестановки генерирует все комбинации параметров в заданном диапазоне и помещает в конфигурационный файл. Функция генерации параметров имеет следующий вид (\ref{lst:autopar}):

\begin{lstlisting}[language=Python, label=lst:autopar, caption=Реализация перемещений муравья]
def form_coefs():
    coefsgen = []
    for i in range(11):
        stack = []
        alpha = round(0.0 + 0.1 * i, 2)
        betta = round(1 - alpha, 2)
        # stack.append([alpha, betta])

        stack.append([round(0.0 + 0.1 * j, 2) for j in range(11)])
        stack.append([3])         # q
        stack.append([0.5])       # init
        stack.append([0.7])       # eps
    return stack
\end{lstlisting}
    \section{Тестирование ПО}
    Результаты тестирования ПО приведены в таблице \ref{table:testing}. 
\begin{table}[!h]
    \centering
    \caption{Тестирование ПО}
    \begin{tabular}{||c|c|c||}
        \hline
        \textit{Матрица} & \textit{Полный перебор} & \textit{Муравьиный алгоритм} \\ \hline\hline
        $\begin{matrix} 0 & 3 & 6 & 1 \\ 1 & 0 & 2 & 1 \\  2 & 1 & 0 & 2 \\ 1 & 2 & 3 & 0 \end{matrix}$ & 6 6 6 6 & 6 6 6 6 \\ \hline
        $ \begin{matrix} 0 & 3 & 7 & 5 & 4 \\ 1 & 0 & 3 & 2 & 1 \\ 4 & 1 & 0 & 1 & 1 \\ 3 & 1 & 2 & 0 & 1 \\ 2 & 1 & 3 & 2 & 0 \end{matrix}$ & 10 10 10 10 10 & 10 10 10 10 10 \\\hline
    \end{tabular}
    \label{table:testing}
\end{table}
\newpage
	\section*{Вывод}
	
	Было написано и протестировано программное обеспечение для решения поставленной задачи.
    	
\newpage