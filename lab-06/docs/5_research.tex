\chapter{Исследовательский раздел}
\label{cha:research}
Раздел содержит описание классов данных для которых проведена параметризация, технические характеристики устройства, на котором проведен эксперимент. Также раздел содержит результаты параметризации алгоритма муравьиной колонии на выбранных классах данных.

\section{Технические характеристики}
Тестирование выполнялось на устройстве со следующими техническими характеристиками:
\begin{itemize}
        \item операционная система: macOS Ventura 13.0;
        \item оперативная память: 8 Gb;
        \item процессор: Apple M1.
\end{itemize}

\section{Постановка эксперимента}
Эксперимент проведен на матрицах \ref{math:class1} и \ref{math:class2} типа int. Количество муравьев фиксировано и равно размеру матрицы смежности. 
Проведенный эксперимент определяет комбинацию параметров, решающих задачу с наилучшим качеством. Качественная оценка работы алгоритма на определенном наборе зависит от двух критериев: количество дней жизни муравьиной колонии и погрешность результата. Для каждого набора параметров поведен один замер. Данные не усреднялись.

После параметризации были проведены временные замеры выполнения алгоритмов. Для каждого набора параметров было проведено 100 замеров.

Во время тестирования устройство было подключено к блоку питания и не нагружено никакими приложениями, кроме встроенных приложений окружения, окружением и системой тестирования. Оптимизация компилятора была отключена.

\section{Класс данных №1}

Класс данных №1 описан согласно следующей матрице смежности (\ref{math:class1}): 

\begin{equation}\label{math:class1}
    \begin{Vmatrix}
        0 & 5 & 8 & 2 & 6 & 9 & 6 & 7 & 8 & 9 \\
        2 & 0 & 7 & 5 & 7 & 8 & 7 & 3 & 4 & 3 \\
        4 & 5 & 0 & 7 & 1 & 3 & 1 & 4 & 2 & 5 \\
        5 & 8 & 9 & 0 & 6 & 9 & 6 & 5 & 8 & 3 \\
        4 & 3 & 3 & 3 & 0 & 3 & 3 & 2 & 7 & 2 \\
        7 & 5 & 5 & 6 & 7 & 0 & 7 & 5 & 8 & 9 \\
        3 & 2 & 3 & 2 & 1 & 3 & 0 & 1 & 2 & 4 \\
        4 & 1 & 6 & 3 & 5 & 2 & 3 & 0 & 3 & 4 \\
        1 & 1 & 2 & 4 & 2 & 5 & 1 & 7 & 0 & 3 \\
        5 & 6 & 1 & 2 & 3 & 4 & 2 & 1 & 1 & 0
    \end{Vmatrix} 
\end{equation}

Результаты работы алгоритма с различными комбинациями представлены в \hyperref[sec:fig1]{приложении А}. Таблица \ref{tab:class1} содержит выборку параметров, решающих задачу с наилучшим, насколько это возможно, качеством.

\begin{table}[!h]
\centering
    \caption{Комбинации параметров}
    \renewcommand{\arraystretch}{1.15}
    \begin{tabular}{||p{0.1\textwidth}p{0.1\textwidth}p{0.1\textwidth}p{0.1\textwidth}p{0.09\textwidth}p{0.09\textwidth}||}
        \hline
        $\alpha$ & $\beta$ & $\rho$ & N & рез. & погр. \\ \hline\hline
        0.0 & 1.0 & 0.9 & 11 & 21 & 0 \\ 
        0.1 & 0.9 & 0.6 & 11 & 21 & 0 \\ 
        0.0 & 1.0 & 0.0 & 11 & 21 & 0 \\ 
        0.3 & 0.7 & 0.0 & 11 & 21 & 0 \\ 
        0.7 & 0.3 & 1.0 & 11 & 21 & 0 \\ 
        0.9 & 0.1 & 0.0 & 11 & 21 & 0 \\ 
        0.9 & 0.1 & 1.0 & 11 & 21 & 0 \\ 
        1.0 & 0.0 & 0.8 & 11 & 21 & 0 \\ 
        0.1 & 0.9 & 0.0 & 12 & 21 & 0 \\ 
        0.8 & 0.2 & 0.0 & 12 & 21 & 0 \\ 
        0.8 & 0.2 & 0.1 & 13 & 21 & 0 \\ 
        0.9 & 0.1 & 0.2 & 13 & 21 & 0 \\ 
        1.0 & 0.0 & 0.4 & 13 & 21 & 0 \\ 
        0.4 & 0.6 & 1.0 & 14 & 21 & 0 \\ 
        0.4 & 0.6 & 0.0 & 15 & 21 & 0 \\ 
        \hline
    \end{tabular}
\label{tab:class1}
\end{table}

\section{Класс данных №2}

Класс данных №1 описан согласно следующей матрице смежности (\ref{math:class2}): 

\begin{equation}\label{math:class2}
    \begin{Vmatrix}
        0 & 5 & 6 & 7 & 8 & 9 & 10 & 11 & 12 & 13 \\
        1 & 0 & 6 & 7 & 8 & 9 & 10 & 11 & 12 & 13 \\
        1 & 2 & 0 & 7 & 8 & 9 & 10 & 11 & 12 & 13 \\
        1 & 2 & 3 & 0 & 8 & 9 & 10 & 11 & 12 & 13 \\
        1 & 2 & 3 & 4 & 0 & 9 & 10 & 11 & 12 & 13 \\
        1 & 2 & 3 & 4 & 5 & 0 & 10 & 11 & 12 & 13 \\
        1 & 2 & 3 & 4 & 5 & 6 & 0 & 11 & 12 & 13 \\
        1 & 2 & 3 & 4 & 5 & 6 & 7 & 0 & 12 & 13 \\
        1 & 2 & 3 & 4 & 5 & 6 & 7 & 8 & 0 & 13 \\
        1 & 2 & 3 & 4 & 5 & 6 & 7 & 8 & 9 & 0 
    \end{Vmatrix}
\end{equation}

Результаты работы алгоритма с различными комбинациями представлены в \hyperref[sec:fig2]{приложении Б}. Таблица \ref{tab:class2} содержит выборку параметров, решающих задачу с наилучшим, насколько это возможно, качеством.

\begin{table}[!h]
    \centering
    \caption{Комбинации параметров}
    \renewcommand{\arraystretch}{1.15}
    \begin{tabular}{||p{0.1\textwidth}p{0.1\textwidth}p{0.1\textwidth}p{0.1\textwidth}p{0.09\textwidth}p{0.09\textwidth}||}
        \hline
        $\alpha$ & $\beta$ & $\rho$ & N & рез. & погр. \\ \hline\hline
        0.2 & 0.8 & 0.6 & 12 & 58 & 0 \\ 
        1.0 & 0.0 & 0.0 & 12 & 58 & 0 \\ 
        0.1 & 0.9 & 0.0 & 13 & 58 & 0 \\ 
        0.9 & 0.1 & 0.0 & 13 & 58 & 0 \\ 
        0.4 & 0.6 & 0.0 & 15 & 58 & 0 \\ 
        0.5 & 0.5 & 1.0 & 15 & 58 & 0 \\ 
        0.7 & 0.3 & 1.0 & 15 & 58 & 0 \\ 
        0.8 & 0.2 & 0.0 & 15 & 58 & 0 \\ 
        0.3 & 0.7 & 0.0 & 16 & 58 & 0 \\ 
        0.6 & 0.4 & 0.0 & 16 & 58 & 0 \\ 
        1.0 & 0.0 & 0.4 & 16 & 58 & 0 \\ 
        0.2 & 0.8 & 0.0 & 17 & 58 & 0 \\ 
        1.0 & 0.0 & 0.5 & 17 & 58 & 0 \\ 
        0.7 & 0.3 & 0.9 & 18 & 58 & 0 \\ 
        1.0 & 0.0 & 0.9 & 19 & 58 & 0 \\ 
        \hline
    \end{tabular}
    \label{tab:class2}
\end{table}

\section{Результаты эксперимента}

Ниже приведена выборка значений, решающих задачу на двух классах данных с наилучшим качеством (\ref{tab:classes-res}):

\begin{table}[!h]
    \centering
    \renewcommand{\arraystretch}{1.2}
    \caption{Комбинации параметров}
    \begin{tabular}{||p{0.1\textwidth}p{0.1\textwidth}p{0.1\textwidth}p{0.05\textwidth}p{0.05\textwidth}p{0.05\textwidth}p{0.05\textwidth}p{0.05\textwidth}p{0.05\textwidth}||}
        \hline
        $\alpha$ & $\beta$ & $\rho$ & \multicolumn{2}{c}{N} & \multicolumn{2}{c}{рез.} & \multicolumn{2}{c||}{погр.} \\
        \hline
        \multicolumn{3}{||c}{класс данных} & 1 & 2 & 1 & 2 & 1 & 2 \\ \hline\hline
        0.3 & 0.7 & 0.0 & 11 & 16 & 21 & 58 & 0 & 0 \\ 
        0.7 & 0.3 & 1.0 & 11 & 15 & 21 & 58 & 0 & 0 \\ 
        0.9 & 0.1 & 0.0 & 11 & 13 & 21 & 58 & 0 & 0 \\ 
        0.1 & 0.9 & 0.0 & 12 & 13 & 21 & 58 & 0 & 0 \\ 
        0.8 & 0.2 & 0.0 & 12 & 15 & 21 & 58 & 0 & 0 \\ 
        0.8 & 0.2 & 0.1 & 13 & 12 & 21 & 58 & 0 & 0 \\ 
        1.0 & 0.0 & 0.4 & 13 & 16 & 21 & 58 & 0 & 0 \\ 
        0.4 & 0.6 & 0.0 & 15 & 15 & 21 & 58 & 0 & 0 \\ 
        \hline
    \end{tabular}
        
    \label{tab:classes-res}
\end{table}

Для временных замеров была взята следующая конфигурация: $\alpha = 0.8$, $\beta = 0.2$, $\rho = 0.1$, $N = 15$. Замеры показали, что на матрицах \ref{math:class1} и \ref{math:class2} алгоритм муравьиной колонии решает поставленную задачу примерно в 400 раз быстрее, чем алгоритм полного перебора. 

\section{Вывод}\label{sec:exp-sum}
Была проведена параметризация алгоритма для двух классов данных. Оптимизация задачи коммивояжера алгоритмом муравьиной колонии показывает следующий результат: при $\alpha = 0.8$, $\beta = 0.2$, $\rho = 0.1$, $N = 15$ поставленная задача будет решена в 400 раз быстрее. 