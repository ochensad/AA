\chapter{Аналитический раздел}
\label{cha:analytical}
    \section{Задача коммивояжера}

    \par Задача коммивояжера (англ. The Travelling Salesman Problem) формулируется следующим образом [\ref{bib:3}]:
	\par «Коммивояжер должен посетить N городов, побывав в каждом из них ровно по одному разу и завершив путешествие в том городе, с которого он начал. В какой последовательности ему нужно обходить города, чтобы общая длина его пути была наименьшей?»
	\par Задача может быть смоделирована на ориентированном взвешенном графе согласно правилам, сформулированным ниже.
	\begin{enumerate}
    	\item Город представлен вершиной графа;
    	\item Пути -- это ребра графа;
    	\item Расстрояние -- это вес ребра графа.
    \end{enumerate}

    \par В целях упрощения задачи, стоит принять тот факт, что моделируемый граф является полностью связным, что обеспечивает наличие пути между двумя любыми произвольными городами. Таким образом, решение задачи -- это нахождение Гамильтонова цикла минимального веса в полном взвешенном графе.

    \par Совершенно очевидно, что задача может быть решена последовательным перебором всех возможных путей и выбором самого оптимального. Такой метод именуется как метод «грубой силы». Для 𝑁 городов существует \begin{math} \frac{(N-1)!}{2} \end{math} различных путей. Факториал является чрезвычайно быстро растущей функцией –- он растёт быстрее, чем любая показательная функция или любая степенная функция. Именно поэтому задача коммивояжёра интересна для тестирования различных алгоритмов.

    \section{Муравьиный алгоритм}
    \subsection*{Принципы поведения муравьиной колонии}

    \par Отдельного муравья нельзя назвать сообразительным существом. Все механизмы, обеспечивающие столь сложное поведение муравьиной колонии, обоснованы формированием роевого интеллекта -- особи, составляющие колонию, исключают централизованное управление, основу их поведения составляет самоорганизация. Самоорганизация, в свою очередь, является результатом взаимодействия четырех компонентов [\ref{bib:4}]:

    \begin{enumerate}
    	\item многократность;
    	\item положительная обратная связь;
    	\item отрицательная обратная связь;
    	\item случайность.
    \end{enumerate}

    \par Муравьи для обмена информацией используют разнесенный во времени тип взаимодействия, при котором один субъект изменяет некоторый параметр окружающей среды, а остальные субъекты используют состояние этого параметра при нахождении в его окрестности [\ref{bib:4}]. Такой контакт называется стигмержи. Биологическим стигмержи для муравьиной колонии является феромон, оставляемый муравьем на земле как след своего перемещения.

    \subsection*{Муравьиный алгоритм в решении задачи коммивояжера}

    \par Алгоритм, основанный на имитации природных механизмов самоорганизации муравьиной колонии, может быть применен в решении задачи коммивояжера. Для описания данного подхода к решению поставленной задачи, следует рассмотреть, каким образом реализуются компоненты социального взаимодействия в рамках выбранной математической модели.

    \begin{enumerate}
    	\item \textbf{Многократность}. Один муравей рассматривается как самостоятельный коммивояжер, решающий задачу. Одновременно несколько муравьев занимаются поиском одного маршрута.
    	\item \textbf{Положительная обратная связь}. Муравей оставлет после перемещения след из феромона и перемещается по следу. Чем больше феромона оставлено на пути -- ребре графа, тем больше муравьев будут перемещаться по этому пути.
    	\item \textbf{Отрицательная обратная связь}. Использование лишь положительной обратной связи может привести к тому, что алгоритм сведется в классический «жадный» алгоритм. Чтобы не допустить подобного случая, феромон имеет свойство испаряться.
    	\item \textbf{Cлучайность}. Вероятностное правило, которое гласит, что вероятность включения ребра графа в маршрут муравья пропорциональна количеству феромона на нем, гарантирует реализацию данного компонента взаимодействия.
    \end{enumerate}

    \subsection*{Параметризация метода}

    \par Параметризация -- это определение таких параметров, надстроек и возможных режимов работы метода, при котором задача будет решена с наилучшим качеством, согласно мерам оценки этого качества. При параметризации метода, основанного на алгоритме муравьиной колонии, вводится понятие муравья. Муравей обладает памятью, представленной списком посещенных за день городов, зрением и виртуальным следом феромона на ребре.

    \par Зрение -- это локальная статистическая величина, выражающая эвристическое желание посетить город $𝑗$ из города $𝑖$ [\ref{bib:4}]. Виртуальный след, в свою очередь, представлет подтвержденное опытом колонии желание посетить город $𝑗$ из города $𝑖$. Вероятностное правило, определяющее вероятность перехода муравья $𝑘$ из города $𝑖$ в город $𝑗$ на итерации $𝑡$ определено согласно \ref{ant-prob}:
	\begin{equation}\label{ant-prob}
		P_{ij, k}\left(t\right) = 
		\begin{cases}
		\frac{
			\displaystyle \left[\tau_{i,j}\left(t\right)\right]^{\alpha} \cdot \left[\eta_{i,j}\left(t\right)\right]^{\beta} 	
		}{
			\displaystyle \sum_{l \in J_{i, k}} \left[\tau_{i,l}\left(t\right)\right]^{\alpha} \cdot \left[\eta_{i,l}\left(t\right)\right]^{\beta} 	
		} & \text{если }j \in J_{i,k}, \\
		
		0 & \text{иначе.}
		
		\end{cases}
	\end{equation}

	\par В формуле \ref{ant-prob} приняты следующие обозначения. Список городов, которые еще не посещены муравьем $k$, который находится в городе $i$ обозначен как $J_{i,k}$, $\eta$ -- видимость, величина, обратная расстоянию между городами $i$ и $j$. Виртуальный след феромона на ребре графа $(i, j)$ на итерации $t$ обозначен как $\tau_{i, j}\left(t\right)$. $\alpha$ -- регулируемый параметр, задающий вес следа феромона, $\beta$ задает видимость при выборе маршрута.

    \par Однако, правило \ref{ant-prob} определяет лишь ширину зоны города $j$ [\ref{bib:5}]: общая зона определяется по принципу "колеса рулетки": каждый город на ней имеет свой сектор с площадью, пропорциональной вероятности \ref{ant-prob}. Для выбора вбрасывается случайное число, и по нему определяется сектор, соответствующий городу.

	\par После завершения маршрута муравей $k$ должен оставить на ребре  $(i, j)$ количество феромона согласно следующему правилу \ref{pheromone}:
	\begin{equation}\label{pheromone}
	\Delta\tau_{ij, k}\left(t\right) = 
	\begin{cases}
	\displaystyle	\frac{Q}{L_k\left(t\right)} & (i, j) \in T_k\left(t\right), \\
	\displaystyle   0 & \text{иначе.}
	\end{cases}
	\end{equation}
	
	\par Здесь $T_k\left(t\right)$ -- маршрут, который был пройден муравьем $k$ за итерацию $t$, $L_k\left(t\right)$ -- длина этого маршрута, а $Q$ -- регулируемый параметр, имеющий значение порядка длины оптимального пути \cite{ant-ulya}.

	\par Испарение феромона обеспечивается правилом \ref{vanish}:
	\begin{equation}\label{vanish}
	\Delta\tau_{ij}\left(t + 1\right) = \left(1 - \rho\right) \cdot \tau_{ij}\left(t\right) + \sum^{m}_{k = 1}\Delta\tau_{ij, k}\left(t\right),
	\end{equation}
	где $m$ -- количество муравьев в колонии.


	\par В начале алгоритма количество феромона принимается равным небольшому положительному числу. Количество муравьев постоянно и равно количеству вершин графа.

	\par Также для улучшения временных характеристик муравьиного алгоритма вводят так называемых элитных муравьев [\ref{bib:4}]. Элитный муравей усиливает ребра наилучшего маршрута, найденного с начала работы алгоритма. Количество феромона, откладываемого на ребрах наилучшего текущего маршрута $T^+$, принимается равным \begin{math} \frac{Q}{L^+} \end{math}, где $L^+$ -- длина маршрута $T^+$. Этот феромон побуждает муравьев к исследованию решений, содержащих несколько ребер наилучшего на данный момент маршрута. Если в муравейнике есть $e$ элитных муравьев, то ребра маршрута $T^+$ будут получать общее усиление
	\begin{equation}\label{elit}
	\Delta\tau_{e} = e \cdot \frac{Q}{L^+}.
	\end{equation}

	\section*{Вывод}
\par Программное обеспечение, решающее поставленную задачу, может работать следующим образом. На вход алгоритму подается матрица смежности и описанные в разделе регулируемые параметры. Программа возвращает длину кратчайшего пути. 

Моделируемый граф является полностью связным, что обеспечивает наличие пути между двумя произвольными городами.

Разрабатываемое программное обеспечение должно реализовывать два алгоритма -- алгоритм муравьиной колонии и алгоритм «грубой силы». Таким образом, можно будет сравнить временные характеристики описанных в разделе методов решения поставленной задачи.

\newpage